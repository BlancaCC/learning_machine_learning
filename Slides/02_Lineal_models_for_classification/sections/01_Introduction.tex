\section{Classification introduction}

\subsection{Problem contextualization}
\begin{frame}
    \frametitle{Goal of classification}

    The goal in classification is to take an input vector $x$ an 
    assign it to one of the the $K \in \N$ discrete classes 
    $\mathcal{C}_k$ where $k \in \{1, \ldots, K\}.$

    Some properties: 

    \begin{itemize}
        \item In the most common scenario, the classes are taken to be disjoint. 
        \item The input space is thereby dividen into \textit{decision regions} whose boundaries are called \textit{decision boundaries} or surfaces. 
    \end{itemize}
    If we consider linear models for classification, the decisión surface are linear functions of the input vector $x \in \R^D$, 
    and hence are defined by $(D-1)$-dimensional hyperplanes.      

\end{frame}

% Página 178
\begin{frame}
    \frametitle{Problem formulation}

    \begin{itemize}
        \item Probabilistic models: $K = 2$, two-class problems where: 
        \begin{itemize}
            \item There is a single target variable 
            $$t \in \{0,1\}$$ 
            \item Class $C_1$ is represented by $t= 1$,
            \item Class $C_2$ is represented by $t= 0$. 
            \item We can interpret the value of $t$ as the probability of class $C_1$ taking only extreme values $\{0,1\}$.
        \end{itemize}

        \item For $K > 2$ classes : \textbf{1 - of- K} coding scheme: 
        $$t \in \{0,1\}^{K} \text{where one only appears one time. }$$
    \end{itemize}
\end{frame}

\begin{frame}
    \frametitle{Distinct appraches to the classification problem}

    \begin{itemize}
        \item Constructing a \textbf{\textit{discriminant function}} that directly assigns each vector $x$ to a specific class. 
        \item Using $p(C_k | x)$ \textit{parametric function}.
        \item Using $p(C_k | x)$ with a generative approach. 
    \end{itemize}
\end{frame}

\subsection{Math formulation}
\begin{frame}
    \frametitle{Math formulation}

    

\end{frame}